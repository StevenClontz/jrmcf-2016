\documentclass{article}[12pt]
%\usepackage{fullpage}
%\usepackage{fullpage}
\usepackage{amsmath}
\usepackage{latexsym}
\usepackage{amssymb,amsfonts}
\usepackage{graphicx}
\usepackage{graphics}
\usepackage[margin=.75in]{geometry}

\graphicspath{ {../../assets/} }

\begin{document}

\large


\newcommand{\activityname}{
  Secret of Nim
}
\newcommand{\subtitle}{
  It's all fun and Game Theory
}

\phantom{.}\hspace{-.5in}\begin{tabular}{lr}
 \begin{tabular}{l}
    \includegraphics[width=2in]{AUExploreLogo.pdf}
 \end{tabular}
 & \hspace{.5in}
 \begin{tabular}{r}
    {\Huge \activityname}
 \end{tabular}
\end{tabular}
\thispagestyle{empty}

\noindent\hrulefill
\phantom{.}\vspace{.15in}

\Large

Math isn't just about equations and shapes. Studying two-player games and determining which player has a \textbf{winning strategy} which cannot be defeated by the opponent is an aspect of \textbf{Game Theory}.

\

Here's a couple of examples. Can you figure out if \textbf{Player 1} or \textbf{Player 2} has a winning strategy?

\

\noindent\textbf{\Huge Take One-or-Two}

\textbf{Setup:} Place 15 tokens on the table.

\textbf{Gameplay:} Players alternate taking one or two tokens from the table.

\textbf{Object:} To take the last token from the table.

\

\noindent\textbf{\Huge Take One-or-Two-or-Three}

Same as Take One-or-Two except players can take three tokens from the table.

\

\noindent\textbf{\Huge Nim}

\textbf{Setup:} Place 5 tokens in 3 rows on the table.

\textbf{Gameplay:} Players alternate taking as many tokens as they like, as long as they are from the same row.

\textbf{Object:} To take the last token from the table.

\input{../../footer.tex}